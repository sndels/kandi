\section{Finnish summary - Suomenkielinen tiivistelmä}

\clearpage

\setcounter{subsection}{-1}
\let\oldsubsection=\thesubsection
\renewcommand{\thesubsection}{\thesection}

\subsection{Kohteenseuranta syvillä neuroverkoilla}

351885 Santeri Salmijärvi (santeri.salmijarvi@aalto.fi)\\
Tekstipaja 2: Henna Juslin 15.8. 12-13:30\\
(Varsinainen käsittelyosa on vielä kesken, joten näyte esittelee taustateoriaa)\\
Kohteenseuranta on aktiivisesti tutkittu konenäön osa-alue, jossa tavoitteena on seurata
osoitettua kohdetta videossa. Sillä on sovelluksia muun muassa videovalvonnassa, koneen ja
ihmisen välisessä viestinnässä sekä lisätyssä todellisuudessa. Luonteeltaan kohteenseurannalla
on yhtäläisyyksiä kuva-analyysisovelluksiin ja täten myös käytetyt menetelmät ovat osittain
samoja. Viime vuosina kohteenseurantaan onkin sovellettu myös syviä neuroverkkoja niiden
menestyttyä kuvaluokittelussa. Tämän työ perehtyy syväoppimiseen, kohteenseurantaan sekä
siihen käytettyihin syviin neuroverkkoihin. Lisäksi työssä käsitellään tehtävään tarkoittettujen
verkkojen vertailuperiaatteita ja käytettyjä aineistoja.

Keinotekoinen neuroverkko on löyhästi aivojen toimintaperiaatteiden mukaan rakennettu
koneoppimismalli, joka mallinnetaan yleisimmin syötteinä, tulosteina ja niitä yhdistävänä
sarjana kerroksia. Kerrokset koostuvat neuroneista, joilla on sarja painotettuja syötteitä,
niiden summalla syötettävä aktivointifunktio ja mahdollinen poikkeutusarvo. Kohteenseurannassa
tyypillinen neuronityyppi on Rectified Linear Unit, joka toteuttaa aktivointifunktiota
$g (z) = \max\{0,z\}$. Se tuottaa hyöydyllisen epälineaarisen muunnoksen syötteistä, mutta
sen yleistyminen ja optimoinnin keveys ovat silti vertailukelpoisia lineaaristen mallien kanssa.
Syötteet ja tulosteet voidaan mallintaa vektoreina ja painot sekä poikkeutukset matriiseina.

Syvä neuroverkko tarkoittaa tyypillisesti neuroverkkoa, jossa välikerroksina on monta
piilotettua kerrosta. Koko verkkoa siis opetetaan tuottamaan syötteistä haluttu tuloste
välittämättä yksittäisen välikerroksen tulosteista. Yleisin kohteenseurannassa käytetty
verkkotyyppi on eteenpäin syöttävä eli ns. MultiLayer Perceptron (MLP). Eteenpäin syöttävässä
verkossa tieto kulkee aina kerroksesta seuraavaan, eikä niiden välillä ole esimerkiksi
takaisinkytkentöjä.

Verkon painot alustetaan tyypillisesti pienillä satunnaisluvuilla ja poikkeutusarvot nollaan
tai pieniin satunnaislukuihin. Yksinkertaisimmillaan koneoppiminen toteutetaan syöttämällä
verkolle tunnettua dataa ja optimoimalla sen painoja vertaamalla tulostetta haluttuun arvoon.
Tulosteen eroa haluttuun arvoon mallinnetaan tappiofunktiolla, jonka gradientti lasketaan
koko verkolle esimerkiksi taaksepäin syöttämällä. Menetelmässä virhearvo syötetään takaisin
pitkin kerroksia ja jokaiselle neuronille asetetaan kontribuutioarvo. Tätä arvoa käytetään
gradientin määrittämiseksi painoille ja kutakin painoa siirretään hieman vastakkaiseen suuntaan
tappiofunktion minimoimiseksi. Prosessia toistetaan uusilla koulutusaineistoilla, kunnes
koulutuksen katsotaan olevan valmis.

Konvoluutiota suoran matriisikertolaskun sijaan vähintään yhdessä kerroksessaan käyttävää
neuroverkkoa kutsutaan konvoluutioneuroverkoksi. Konvoluutio on kahdelle funktiolle suoritettava
operaatio, joka kuvataan yleisesti niiden pistetulojen integraalina toisen funktion
siirtämisen funktiona. Kuva-analyysissä konvoluutio tapahtuu käsittelemällä syötekuvaa
siirrettävällä kernelillä eli matriisilla painoarvoja. Kukin neuroni käsittelee vain
kernelinsä rajoittamaa osaa syötteestä, mutta konvoluutiokerrosten ketjuttamisen myötä
syvemmät kerrokset ovat epäsuorassa yhteydessä koko verkon syötteeseen tai suurimpaan
osaan siitä.

Konvoluutioverkon kerros muodostuu kolmesta vaiheesta: konvoluutio, tunnistus ja kerääminen.
Ne voidaan mallintaa verkkoon erillisinä kerroksina mahdollistaen modulaarisen rakenteen.
Konvoluutiovaiheessa syöte käsitellään kerneliä liikuttaen. Tunnistusvaiheessa suoritetaan
useita konvoluutioita rinnakkain ja kukin niistä syötetään lineaariseen aktivointifunktioon.
Nämä tulosteet syötetään tunnistusvaiheessa epälineaariseen aktivointifunktioon ja
keräämisvaiheessa lähekkäiset tunnistuksen tulosteet yhdistetään keräysfunktiolla.

Kohteenseurantaa videossa on tutkittu vuosikymmeniä. Tyypillisesti seuranta tapahtuu videosta,
jonka ensimmäisessä kuvassa on osoitettu haluttu seurannan kohde, mutta joissain tapauksissa
sovelluksen on löydettävä se kuvauksen perusteella. Tehtävä ei ole laajasta tutkimuksesta
huolimatta ratkaistu ja seurannan epäonnistumista aiheuttavat edelleen hankalat olosuhteet
kuten liike-epäterävyys tai kohteen hetkellinen poistuminen kuvasta.

Merkittävät varhaiset sovellukset hyödynsivät kohteesta otettuja kuvasarjoja kohteen etsimiseksi
tai esittivät kohteen sen rajat määrittävänä käyränä. Modernit seurantamenetelmät voidaan jakaa
karkeasti generatiivisiin ja erotteleviin, mutta niiden yhdistelmiä on myös tutkittu. Generatiiviset
sovellukset etsivät kuvasta parhaiten kohteesta muodostettua mallia vastaavia alueita ja erottelevat
toteuttavat seurannan erottamalla kohteen taustasta esimerkiksi soveltamalla luokitteluverkkoa
pienempiin osiin kuvasta.

Koulutukseen käytetyt aineistot ovat yhtä tärkeitä kuin itse verkon rakenne. Seurantaverkkoa
kehitettäessä on tavallista esikouluttaa ominaisuuksia erottelevat kerrokset yksi kerrallaan
esimerkiksi luokitteluun tarkoitettujen käsin merkittyjen tai hakusanalla kuvapalveluista
ladattujen kuva-aineistojen avulla parempien tulosten takaamiseksi. Yksi merkittävimmistä
käytetyistä aineistoista on vuosittaisiin ImageNet Large Scale Visual Recognition Challenge
-haasteisiin päivitettävä kuvakanta.

Yksittäisten kuvien lisäksi koulutukseen käytetään kohteenseurantaan keskittyneitä
videoaineistoja, joilla voidaan arvioida verkon ehdottamaa kohteen sijaintia todelliseen.
Merkittäviä seurantavideoita sisältäviä aineistoja ovat TB-100 ja sen alajoukko TB-50 sekä
vuoteen 2012 järjestetyn Visual Object Tracking challenge (VOT) -kilpailun aineistot.

Kehitettyjä seurantasovelluksia arvioidaan vertaamalla niiden tuloksia aiemmin julkaistuihin
samaa aineistoa käyttämällä. Yksi käytetyistä aineistoista on Visual Tracker Benchmarkin kokoama
TB-50, johon on julkaistu sen tekohetkellä merkittävien vapaasti saatavilla olleiden sovellusten
tulokset vertailua varten. Myös VOT:n kaltaisten kilpailujen tuloksia ja aineistoja voidaan
käyttää myöhempien menetelmien arviointiin.

\renewcommand{\thesubsection}{\oldsubection}
