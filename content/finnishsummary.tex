\section{Finnish summary - Suomenkielinen tiivistelmä}

\clearpage

\setcounter{subsection}{-1}
\let\oldsubsection=\thesubsection
\renewcommand{\thesubsection}{\thesection}

\subsection{Kohteenseuranta syvillä neuroverkoilla}

Kohteenseuranta on aktiivisesti tutkittu konenäön osa-alue, jossa tavoitteena on seurata
osoitettua kohdetta videossa. Luonteeltaan sillä on yhtäläisyyksiä kuva-analyysisovelluksiin
ja täten myös käytetyt menetelmät ovat osittain samoja. Viime vuosina kohteenseurantaan onkin
sovellettu myös syviä neuroverkkoja niiden menestyttyä kuvaluokittelussa. Tämän työ perehtyy
syväoppimiseen, kohteenseurantaan sekä siihen käytettyihin syviin neuroverkkoihin. Lisäksi
työssä käsitellään tehtävään tarkoittettujen verkkojen vertailuperiaatteita ja käytettyjä
aineistoja.

Keinotekoinen neuroverkko on löyhästi aivojen toimintaperiaatteiden mukaan rakennettu 
koneoppimismalli, joka mallinnetaan yleisimmin syötteinä, tulosteina ja niitä yhdistävänä
sarjana kerroksia. Kerrokset koostuvat neuroneista, joilla on sarja painotettuja syötteitä,
niiden summalla syötettävä aktivointifunktio ja mahdollinen painoarvo. Yksinkertaisimmillaan
koneoppiminen toteutetaan syöttämällä verkolle tunnettua dataa ja optimoimalla sen painoja
vertaamalla tulostetta haluttuun arvoon. 

Syvä neuroverkko tarkoittaa tyypillisesti neuroverkkoa, jossa välikerroksina on monta
piilotettua kerrosta. Koko verkkoa siis opetetaan tuottamaan syötteistä haluttu tuloste
välittämättä yksittäisen välikerroksen tulosteista. Yleisin kohteenseurannassa käytetty
verkkotyyppi on eteenpäinsyöttävä eli ns. MultiLayer Perceptron (MLP). Eteenpäinsyöttävässä
verkossa tieto kulkee aina kerroksesta seuraavaan, eikä niiden välillä ole esimerkiksi
takaisinkytkentöjä.

\renewcommand{\thesubsection}{\oldsubsection}
