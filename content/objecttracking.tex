\section{Object tracking}
Object tracking in video sequences has been researched for decades and \ac{nn}-based
algorithms have recently been researched \authorcomment{cite works here} as \ac{nn}s'
capabilities in classification tasks have been noted \authorcomment{cite relevant work(s)}.
There has also been development in the area of benchmarking the tracking algorithms
\cite{OT_BENCH}.

\subsection{Overview}
Tracking methods can be roughly divided to generative and discriminative, but combinations
of them have also been proposed.

Generative methods search the frame for the best matches
to a template of an appearance model of the subject. Template methods based on pixel
intensity and color histograms perform well with no drastic changes in object appearance
and non-cluttered backgrounds. Appearance models learnt from training can be less affected
by appearance variations and adaptive schemes provide added flexibility, while sparse
models handle occlusion and image noise better.~\cite{OBJECT_PLS} \authorcomment{add citations}

Discriminative methods consider tracking as a binary classification problem. They take
the background also into account to separate the target from it. One approach is to sample
the target's surroundings in the frame to produce a model of the background.

\subsection{Deep neural networks in tracking}
Overview of task from \ac{dnn}-point of view, strengths and weaknesses compared to
more traditional solutions.

\subsection{Data sets and evaluation}
Overview of the data sets used for training and analysis. Methods used for comparing
performance.
