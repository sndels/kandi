\section{Object tracking}
Object tracking in video sequences has been researched for decades using different
approaches for defining the target and adapting to changes in its shape or orientation.
The situations most likely to cause tracking failure have also been identified.

\subsection{Target representation}
Tracking methods can be roughly divided to generative and discriminative, but combinations
of them have also been proposed.\authorcomment{cite a combination}

Generative methods search the frame for the best matches to a template of an appearance
model of the subject. Template methods based on pixel intensity and color histograms
perform well with no drastic changes in object appearance and non-cluttered backgrounds.
Appearance models learned from training can be less affected by appearance variations
and adaptive schemes provide added flexibility, while sparse models handle occlusion
and image noise better.~\cite{OBJECT_PLS}
\authorcomment{add citations}

Discriminative methods consider tracking as a binary classification problem. They take
the background also into account to separate the target from it. Used approaches
include refining the initial guess with a support vector machine~\cite{SVT} or utilizing
a relevance vector machine~\cite{SPARSE_BAYESIAN}.

\subsection{Model update}
(Describe roughly the idea of updating the model online, motivations)

\subsection{Challenges}
(Present challenging situations for trackers, motivation for development of better
methods.)
