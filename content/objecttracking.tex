\section{Object tracking}
Object tracking in video sequences has been researched for decades and \ac{nn}-based
algorithms have recently been researched \authorcomment{cite works here} as \ac{nn}s'
capabilities in classification tasks have been noted \authorcomment{cite relevant works}.
There has also been development in the area of benchmarking the tracking algorithms
~\cite{OT_BENCH}.

\subsection{Overview}
Tracking methods can be roughly divided to generative and discriminative, but combinations
of them have also been proposed.

Generative methods search the frame for the best matches
to a template of an appearance model of the subject. Template methods based on pixel
intensity and color histograms perform well with no drastic changes in object appearance
and non-cluttered backgrounds. Appearance models learnt from training can be less affected
by appearance variations and adaptive schemes provide added flexibility, while sparse
models handle occlusion and image noise better.~\cite{OBJECT_PLS}
\authorcomment{add citations}

Discriminative methods consider tracking as a binary classification problem. They take
the background also into account to separate the target from it. Used approaches
include refining the initial guess with a support vector machine~\cite{SVT} or utilizing
a relevance vector machine~\cite{SPARSE_BAYESIAN}.
\authorcomment{go into greater detail about the methods}

\subsection{Deep neural networks in tracking}
(Overview of task from \ac{dnn}-point of view, strengths and weaknesses compared to
more traditional solutions.)

An early implementation of a \ac{cnn}-based tracker~\cite{HUMAN_CNN} pre-dates the work
of Krizhevsky et.~al.~\cite{NIPS_IMAGENET}. It takes modifies the architecture used for
detection to make the network less affected by sifts in the objects position in the
frame. Shift-invariancy is a non-desirable quality in tracking while using previous
positions as a as it might result to mixups with objects similar to the target.~\cite{HUMAN_CNN}

\subsection{Data sets and evaluation}
Overview of the data sets used for training and analysis. Methods used for comparing
performance.
