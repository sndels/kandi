\section{Data sets and evaluation}
\authorcomment{Overview of the data sets used for training and analysis. Methods
used for comparing performance.}
\authorcomment{Move these to section 3?}

\subsection{Data sets}
The datasets used for training are equally important as the actual network design
\authorcomment{citation}.
Research on networks working with image data has been made easier by larger sets of both
hand-labeled sets and ones obtained by simple keyword searches from online image services.
These kinds of sets can be used to pre-train useful target features to tracking networks.

VOC was a yearly competition for object recognition and VOC 2012 \cite{VOC12} is the last challenge
in the series. The datasets of the challenges are still used for pre-training features
for detection stages in tracking networks. There are four major subsets of hand-labeled
VOC data: classification, segmentation action classification, person layout.
Classification datasets consist of images annotated with the objects contained and
bounding boxes for the objects drawn in the image itself while image segmentation sets
provide additional mask images of the objects and classes in each shot. Action classification
sets contain descriptions and bounding boxes of actions the subjects are performing and
person layout sets contain bounding boxes for the subjects head, hands and feet.
\authorcomment{image examples of the datasets, descriptions to annotations}

The \ac{ilsvrc} \cite{ILSVRC15} is another recognition challenge running since 2010. The
most recent dataset consists of subsets of object localization, object detection and
object detection from video. The last subset is especially beneficial object tracking
tasks as it provides data for training on actual tracking data. The other two sets are also
substantially larger than the respective VOC sets as their labeling has been crowd sourced. 

There has also been an increase in resources solely devoted to tracking data with the
TR-100 -set~\cite{VTB} being a good example. It contains a hundred tracking sequences
with reference positions for the target on each frame. Because some of the targets are
similar or less challenging, a subset of 50 sequences considered challenging is also
provided as TR-50.~\cite{OT_BENCH}
