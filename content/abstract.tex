\keywords{object tracking, deep neural networks, deep learning}
\degreeprogram{Bachelor's Programme in Electrical Engineering}

\begin{abstractpage}[english]
Object tracking is a subsection of computer vision where the aim is to follow a target through a video. The problem has been researched for decades and its use cases include surveillance, human-computer interaction and augmented reality. Traditional methods have provided good results in simple conditions but lost accuracy due to distractors like motion blur and drastic illumination changes motivates the development of more robust trackers.

Recently, deep learning has garnered interest as it has shown promising results in image classification. Object tracking has also been considered as a possible application for deep architectures and this thesis studies the deep neural networks used in trackers as well as the reasoning behind using such solutions. Trackers based on deep neural networks have been researched for their ability to utilize hierarchies of features learned from training. These generic sets of features are seen as a possible way to tackle the difficulties rising form tracking in challenging conditions. Deep networks have also shown good generalization when subjected to new target classes, which improves upon the traditional methods' requirement of hand tuning and domain specific knowledge. Based on the research done in the field, deep neural networks are able to compete with traditional tracking methods.
\end{abstractpage}

\newpage
\thesistitle{Kohteenseuranta syvillä neuroverkoilla}
\keywords{kohteenseuranta, syvät neuroverkot, syväoppiminen}
\degreeprogram{Sähkötekniikan kandidaattiohjelma}
\supervisor{TkT Pekka Forsman}
\advisor{DI Mikko Vihlman}

\begin{abstractpage}[finnish]
Kohteenseuranta on konenäön osa-alue, jossa osoitettua kohdetta seurataan videossa. Ongelmaa on tutkittu vuosikymmeniä ja sille on käyttökohteita muun muassa valvontasovelluksissa, tietokoneen ja ihmisen vuorovaikutuksessa sekä lisätyssä todellisuudessa. Perinteiset menetelmät ovat kyenneet hyviin tuloksiin yksinertaisissa tilanteissa, mutta seurannan heikentyminen liike-epäterävyyden ja suurien valotuksen vaihteluiden kaltaisten tekijöiden myötä motivoi vakaampien sovellusten kehittämistä.

Syväoppiminen on kerännyt viime aikoina mielenkiintoa, sillä se on osoittanut lupaavia tuloksia kuvaluokittelussa. Myös kohteenseurantaa on pidetty mahdollisena syvien arkkitehtuurien sovelluskohteena, minkä innoittamana tämä työ tutkii kohteenseurantaan käytettyjä syviä neuroverkkoja ja perusteita niiden käytölle. Syviin neuroverkkoihin perustuvia seurantasovelluksia on tutkittu, koska syvät verkot oppivat eristämään hierarkisia piirteitä. Näiden yleisten piirteiden käyttö nähdään mahdollisena ratkaisuna haastavien olosuhteiden tuottamiin onglemiin. Syvien verkkojen on havaittu myös yleistyvän hyvin ennalta tuntemattomiin kohdeluokkiin, mikä on parannus perinteisten menetelmien vaatimaan hienosäätöön ja luokkakohtaiseen tietoon. Alueella tehdyn tutkimuksen perusteella syvät neuroverkot pystyvät kilpailemaan perinteisten seurantamenetelmien kanssa.
\end{abstractpage}
