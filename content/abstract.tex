\keywords{keywords in english}
\degreeprogram{Bachelor's Program in Electrical Engineering}

\begin{abstractpage}[english]
Object tracking is a subsection of computer vision, where a target is followed through
a video. The problem has been researched for decades and it's use cases include
surveillance human-computer interaction and augmented reality. Traditional methods
have provided good results in simple conditions but lost accuracy due to distractors
like motion blur and drastic illumination changes motivates the development of more
robust trackers.

Recently, deep learning has garnered interest as it has shown promising results in
image classification. Trackers based on deep neural networks have been researched
for their ability to utilize hierarchies of features learned from training. Generic
sets of features are seen as a possible way of tackling the difficulties of tracking
in challenging conditions. Trackers built on deep neural networks have also shown
good generalization when subjected to new target classes, which improves upon the
traditional methods' requirement of hand tuning and domain specific knowledge. Based
on the research done in the field, deep neural networks are capable of results
comparable to or even better than those of traditional methods.
\end{abstractpage}

\newpage
\thesistitle{Kohteenseuranta syvillä neuroverkoilla}
\keywords{avainsanat suomeksi}
\degreeprogram{Sähkötekniikan kandidaattiohjelma}
\supervisor{TkT Pekka Forsman}
\advisor{DI Mikko Vihlman}

\begin{abstractpage}[finnish]
Kohteenseuranta on konenäön osa-alue, jossa kohdetta seurataan läpi videon. Ongelmaa
on tutkittu vuosikymmeniä ja sille on käyttökohteita muun muassa valvontasovelluksissa,
tietokoneen ja ihmisen vuorovaikutuksessa sekä lisätyssä todellisuudessa. Perinteiset
menetelmät ovat kyenneet hyviin tuloksiin yksinertaisissa tilanteissa, mutta seurannan
heikentyminen liike-epäterävyyden ja suurien valotuksen vaihteluiden kaltaisten
tekijöiden myötä motivoi vakaampien sovellusten kehittämistä.

Viime vuosina syväoppiminen on kerännyt huomiota näytettyään lupaavia tuloksia
kuvanluokittelusovelluksissa. Syviin neuroverkkoihin perustuvia seurantasovelluksia
on tutkittu niiden opetuksesta oppimien hierarkisten piirteiden ansiosta. Nämä yleiset
piirteet nähdään mahdollisena tapana ratkaista hankalien olosuhteiden tuomat haasteet
seurannassa. Syvillä verkoilla toimivat sovellukset ovat myös osoittaneet yleistyvänsä
hyvin uusiin kohdeluokkiin, mikä on parannus perinteisten menetelmien kaipaamaan
hienosäätöön ja luokkakohtaiseen tietoon. Alalla tehdyn tutkimuksen perusteella
syvät neuroverkot pystyvät perinteisten menetelmien kanssa vertailukelpoisiin tai
jopa parempiin tuloksiin.
\end{abstractpage}
