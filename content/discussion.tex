
\section{Discussion}
A noticeable trend since the work of Fan et.~al. (\ref{ssec:early}) is the use of branching in the network to extract different levels of features from sequences (\ref{sssec:mdn},~\ref{sssec:fcn}) or combine multiple low-level image cues (\ref{sssec:deeptrack}). Multiple levels of features were used to have access both to general features and fine detail in discerning the target, while multiple cues helped helped finding more kinds of relations. Another novel use of branching was seen in the work of Nam and Han (\ref{sssec:mdn}), where a unique final classification layer was trained for each tracking sequence. This approach was used to teach the feature extraction layers generic features across all target classes.

Recent work has used both \ac{cnn} (\ref{sssec:learned_hierarch},\ref{sssec:deeptrack},~\ref{sssec:mdn},~\ref{sssec:fcn}) and \ac{sdae} (\ref{sssec:blur}) based methods for feature extraction but \ac{cnn} is clearly the dominant approach. This is likely due to the wide use of them in classification and detection tasks since their good performance in the ImageNet challenge~\cite{NIPS_IMAGENET}. Many trackers use existing classification networks in their feature extraction modules (\ref{sssec:learned_hierarch},~\ref{sssec:fcn}) so the use of \ac{cnn}s carries over from classification research that way as well. Pre-training of especially the feature extraction layers is common as is the case in image classification networks, but learning the needed feature hierarchies exclusively online from the tracking sequence was explored with good results in DeepTrack (\ref{sssec:deeptrack}).

The research on using \ac{dnn}s in trackers is still very young and the number of publications is relatively small. Good comparisons have not been made between many of the more recent methods and the publications themselves have contained benchmarks on different sets of test sequences or slighty differing performance metrics. While certainly interesting, the process of benchmarking the recent methods is beyond the scope of this thesis because it could require modifying the trackers to work with common input and output~\cite{OT_BENCH}. All the publications did themselves present evaluation results to show improvements over other methods on the selected test sets. Tracking speed fell mostly in the area of only a few frames/s with only one tracker claiming a speed of up to 10 frames/s (\ref{sssec:blur}). It should be noted however that the implementations were not said to be optimized and one of them ran on a single CPU thread (\ref{sssec:learned_hierarch}) so advanced optimization techniques and the utilization of a GPU could result in real-time tracking being more viable.
