\section{Introduction}
Object tracking is a large and actively researched sub-area of computer vision. The
main task for a tracker is to indicate the desired subject in a sequence of images.
It can be applied in areas such as human-computer interaction and augmented reality
and is related to other image analysis tasks. In the recent years, use of deep neural
networks has been researched for object tracking following their adoption in image
classification.

Tracking implementations with deep neural networks are investigated because traditional
hand-crafted feature models can suffer from drift or tracking failure in challenging
circumstances like cluttered backgrounds or large changes illumination. Another issue
that hand-crafted features don’t generalize well to new target classes, which is one
of the observed strengths of the hierarchical features a deep learning model acquires
from training. Deep networks are also becoming more viable in real-time tracking as
the computational power that is available increases and more efficient specializations
of deep feature models are developed for the task. The training of deep neural networks
requires a large amount of training data and their development has been made easier by
an increase in the size of applicable datasets.

This thesis explores object trackers that have used deep neural networks in their
architecture. First, deep neural networks and object tracking are familiarized in
enough detail to understand the reasoning and designs used for the trackers utilizing
deep learning. Datasets for training and evaluating trackers are also briefly discussed
as they are important to the research. The main research questions are the following:
what kinds of tracker architectures have used deep neural networks, how deep features
benefit the trackers in question and what kinds of drawbacks emerge from their use? The
work will be done purely as a literature study.
