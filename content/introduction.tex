\section{Introduction}

Object tracking is a large and actively researched sub-area of computer vision. The
main task for a tracker is to find and follow the desired subject in a sequence of
images. Object tracking is closely related to other image analysis tasks so the
implementations also share elements.

The field of image classification took a leap forward in 2012, when
Krizhevsky et.\ al.\ presented record performance in the ImageNet-classification
challenge using a convolutional network. Previous work had dismissed the network
type as unfit for the task.~\cite{NIPS_IMAGENET} Since then, research has shifted
to using convolutional networks as they have several clear advantages over
other network types when used on picture analysis.
\authorcomment{go into benefits and/or give a source for the claim}

With the adoption of convolutional networks, much of the research revolves around
deep neural networks. They consist of visible input and output layers with several
so-called hidden layers in between them. The training of deep neural networks
requires a large amount of training data and their development has been made easier
by an increase in the size of applicaple datasets.

This thesis will present the architectures and principles currently used in deep
neural networks tailored to object tracking tasks. The practices behind training
and evaluating such networks are also introduced.
