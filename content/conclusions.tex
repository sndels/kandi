\section{Conclusions}

This thesis first presented the basic theory of object tracking and \ac{dnn}s to provide basis for understanding the motivations in applying \ac{dnn} architectures to tracking. Several novel and successful trackers were then reviewed with an emphasis on their architectures and the reasoning used in design. Common advantages and drawbacks compared to traditional tracking methods were also noted.

\ac{dnn}s are seen as a potential alternative to hand-crafted low-level models in object tracking. The traditional methods work well in well controlled environments but can have difficulties tracking in challenging conditions like partial occlusion of the target or motion blur. The hiearchical generic features a \ac{dnn} learns are more resilient against such challenges which is a common motivation for developing trackers that use \ac{dnn}s. Learning feature representations from training is also seen as a potential alternative to hand-tuning a traditional model that typically requires domain specific knowledge.~\cite{DLT}

The research of \ac{dnn} based object tracking is still young and trackers have been proposed with emphasis on different challenges. Better generalization and being less affected by challenging conditions are common motivations for choosing deep features over traditional hand-crafted ones and trackers using \ac{dnn}s have shown competitive performance when put against traditional trackers \cite{DEEPTRACK}. However, most \ac{dnn} trackers require pre-training and real-time performance is still difficult to attain even with the advance in computing power.
