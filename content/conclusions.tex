\section{Conclusions}

This thesis first presented the basic theory of \ac{dnn}s and object tracking to
provide basis for understanding the motivations to apply \ac{dnn} architectures
to tracking. Several novel and succesful trackers were then reviewed with an emphasis
on the high-level architectures and reasoning used in developing them. Common advantages
and issues compared to traditional tracking methods were also noted.

\ac{dnn}s are seen as a potential alternative to hand-crafted low-level models in
object tracking. The traditional methods work well in well controlled environments
but can have difficulties tracking in difficult conditions like partial occlusion
of the target or motion blur. The hiearchial generic features a \ac{dnn} learns
are more resilient against such challenges.

%very rough sketch
The research on using \ac{dnn}s in trackers is still very young as even the earliest
example was published only five years ago. The number of publications is also relatively
small and exhaustive comparisons have not been made between many of the proposed
methods [check most recent publications to verify].