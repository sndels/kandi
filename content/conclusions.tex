\section{Conclusions}

This thesis first presented the basic theory of \ac{dnn}s and object tracking to
provide basis for understanding the motivations to apply \ac{dnn} architectures
to tracking. Several novel and successful trackers were then reviewed with an emphasis
on their architectures and the reasoning used in the design. Common advantages and
drawbacks compared to traditional tracking methods were also noted.

\ac{dnn}s are seen as a potential alternative to hand-crafted low-level models in
object tracking. The traditional methods work well in well controlled environments
but can have difficulties tracking in difficult conditions like partial occlusion
of the target or motion blur. The hiearchical generic features a \ac{dnn} learns
are more resilient against such challenges which is a common motivation for developing
a tracker using \ac{dnn}s. Learning feature representations from training is also seen
as a potential alternative to hand-tuning a traditional model which typically requires
domain specific knowledge.~\cite{DLT}

Recent work has used both \ac{cnn}~\cite{DEEPTRACK}~\cite{MDNET}~\cite{VGG} and
\ac{sdae}~\cite{BLUR_TRACK} based methods for feature extraction but \ac{cnn} is
clearly the dominant approach. This is likely due to the wide use of them in classification
and detection tasks since their good performance in the ImageNet challenge~\cite{NIPS_IMAGENET}.
A noticeable trend in some trackers was the use of branching in the network to extract
different levels of features from sequences~\cite{MDNET}~\cite{FCN_TRACK_2}.

The research on using \ac{dnn}s in trackers is still very young and The number of
publications is relatively small. Good comparisons have not been made between many
of the more recent methods and the publications themselves have contained benchmarks
on different sets of test sequences or slighty differing representations of performance.
While certainly interesting, the process of benchmarking is beyond the scope of this
thesis because it could require modifying the trackers to work with input and output~\cite{OT_BENCH}.
All the publications did themselves present evaluation results to show improvements
over other methods on the selected test sets. Tracking speed fell mostly in the area
of only a few frames/s with only one tracker claiming a speed of up to 10 frames/s~\cite{BLUR_TRACK}.
It should be noted that the implementations were not said to be optimized and some of
them only ran on the CPU~\cite{LEARNED_HIERARCH} so advanced optimization techniques and
the utilization of a GPU could result in more viable real-time tracking.

In conclusion, the research of \ac{dnn} based object tracking is still relatively
young and trackers have been proposed with emphasis on different challenges. Better
generalization and being less affected by challenging conditions are common motivations
for choosing deep features over traditional hand-crafted ones and trackers using
\ac{dnn}s have shown competitive performance when put against traditional trackers.
However, real-time performance is still difficult to attain even with the advance
in the hardware available.
