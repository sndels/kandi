\documentclass[english,12pt,a4paper,pdftex,elec,utf8]{aaltothesis}

%% Kirjoita y.o. \documentclass optioiksi
%% korkeakoulusi näistä: arts, biz, chem, elec, eng, sci
%% editorisi käyttämä merkkikoodaustapa: utf8, latin1

%% Käytä näitä, jos kirjoitat englanniksi. Katso englanninokset tiedostosta
%% thesistemplate.tex.
%\documentclass[english,12pt,a4paper,pdftex,elec,utf8]{aaltothesis}
%\documentclass[english,12pt,a4paper,dvips]{aaltothesis}

\usepackage[english,finnish]{babel}
\usepackage[fixlanguage]{babelbib}
\usepackage{cite}
\usepackage[section]{placeins} %prevent figures from floating allover
\usepackage{cleveref} % for cref
\usepackage{booktabs} % for midrule etc.
\usepackage{multirow} % for multirow table
\usepackage{siunitx} % for \ang
\usepackage{float}
\usepackage{subfig} % for subfigures
\usepackage[toc,page]{appendix}
\usepackage[yyyymmdd,hhmmss]{datetime}

%% Symbolit ja lyhenteet
\usepackage{acro} % for acronyms
\acsetup{
  only-used   = true
}

\DeclareAcronym{dnn}{
    short = DNN,
    long  = Deep Neural Network,
    class = abbrev
}

\DeclareAcronym{nn}{
    short = NN,
    long  = Neural Network,
    class = abbrev
}

\DeclareAcronym{mlp}{
    short = MLP,
    long  = MultiLayer Perceptron,
    class = abbrev
}

\DeclareAcronym{relu}{
    short = ReLU,
    long  = Rectified Linear Unit,
    class = abbrev
}

\DeclareAcronym{cnn}{
    short = CNN,
    long  = Convolutional Neural Network,
    class = abbrev
}


%% Aloita uudet kappaleet uudelta sivulta
\usepackage{titlesec}
\newcommand{\sectionbreak}{\clearpage}
\usepackage{graphicx}

\usepackage[numbers]{natbib}

\selectbiblanguage{english}
\bibliographystyle{babunsrt}

%% Matematiikan fontteja, symboleja ja muotoiluja lisää, näitä tarvitaan usein 
\usepackage{amsfonts,amssymb,amsbsy,amsmath}

%% Saat pdf-tiedoston viittaukset ja linkit kuntoon seuraavalla paketilla.
\usepackage[pdfa]{hyperref}
\hypersetup{pdfpagemode=UseNone, pdfstartview=FitH,
  colorlinks=true,urlcolor=red,linkcolor=blue,citecolor=black,
  pdftitle={Default Title, Modify},pdfauthor={Your Name},
  pdfkeywords={Modify keywords}}

%% Shows comments in boxes
\newcommand{\authorcomment}[1]{\fbox{\begin{minipage}{\textwidth}Comment by author:\\#1\end{minipage}}}
\newcommand{\advisorcomment}[1]{\fbox{\begin{minipage}{\textwidth}Comment by advisor:\\#1\end{minipage}}}

%% Hides comments
%\newcommand{\authorcomment}[1]{}
%\newcommand{\advisorcomment}[1]{}

%% Kaikki mikä paperille tulostuu, on tämän jälkeen
\begin{document}

%% Kansilehti
%% Korjaa vastaamaan korkeakouluasi, jos automaattisesti asetettu nimi on 
%% virheellinen 
%%
%% Change the school field to specify your school if the automatically 
%% set name is wrong
% \university{aalto-yliopisto}
% \school{Sähkötekniikan korkeakoulu}

%% VAIN DI/M.Sc.- JA LISENSIAATINTYÖLLE: valitse laitos,
%% professuuri ja sen professuurikoodi.
%%
%\department{Radiotieteen ja -tekniikan laitos}
%\professorship{Piiriteoria}
%%

%% Valitse yksi näistä kolmesta
%%
\univdegree{BSc}
%\univdegree{MSc}
%\univdegree{Lic}

%% Oma nimi
%%
\author{Santeri Salmijärvi}

%% Opinnäytteen otsikko tulee tähän ja uudelleen englannin- tai 
%% ruostinkielisen abstraktin yhteydessä. Älä tavuta otsikkoa ja
%% vältä liian pitkää otsikkotekstiä. Jos latex ryhmittelee otsikon
%% huonosti, voit joutua pakottamaan rivinvaihdon \\ kontrollimerkillä.
%% Muista että otsikkoja ei tavuteta! 
%% Jos otsikossa on ja-sana, se ei jää rivin viimeiseksi sanaksi 
%% vaan aloittaa uuden rivin.
%% 
\thesistitle{Object tracking with deep neural networks}

\place{Espoo}

%% Kandidaatintyön päivämäärä on sen esityspäivämäärä! 
%% 
%TODO
\date{Work in progress! Compiled: \currenttime \space \today}
%\date{27.11.2016}

\supervisor{D.Sc. (Tech) Pekka Forsman}
\advisor{M.Sc. (Tech) Mikko Vihlman}

%% Aaltologo: syntaksi:
%% \uselogo{aaltoRed|aaltoBlue|aaltoYellow|aaltoGray|aaltoGrayScale}{?|!|''}
%% Logon kieli on sama kuin dokumentin kieli
%%
\uselogo{aaltoRed}{''}

%% Tehdään kansilehti
%%
\makecoverpage{}


%% Tiivistelmä
\keywords{object tracking, deep neural networks, deep learning}
\degreeprogram{Bachelor's Programme in Electrical Engineering}

\begin{abstractpage}[english]
Object tracking is a subsection of computer vision where the aim is to follow a target through a video. The problem has been researched for decades and its use cases include surveillance, human-computer interaction and augmented reality. Traditional methods have provided good results in simple conditions but lost accuracy due to distractors like motion blur and drastic illumination changes motivates the development of more robust trackers.

Recently, deep learning has garnered interest as it has shown promising results in image classification. Object tracking has also been considered as a possible application for deep architectures and this thesis studies the deep neural networks used in trackers as well as the reasoning behind using such solutions. Trackers based on deep neural networks have been researched for their ability to utilize hierarchies of features learned from training. These generic sets of features are seen as a possible way to tackle the difficulties rising form tracking in challenging conditions. Deep networks have also shown good generalization when subjected to new target classes, which improves upon the traditional methods' requirement of hand tuning and domain specific knowledge. Based on the research done in the field, deep neural networks are able to compete with traditional tracking methods.
\end{abstractpage}

\newpage
\thesistitle{Kohteenseuranta syvillä neuroverkoilla}
\keywords{kohteenseuranta, syvät neuroverkot, syväoppiminen}
\degreeprogram{Sähkötekniikan kandidaattiohjelma}
\supervisor{TkT Pekka Forsman}
\advisor{DI Mikko Vihlman}

\begin{abstractpage}[finnish]
Kohteenseuranta on konenäön osa-alue, jossa osoitettua kohdetta seurataan videossa. Ongelmaa on tutkittu vuosikymmeniä ja sille on käyttökohteita muun muassa valvontasovelluksissa, tietokoneen ja ihmisen vuorovaikutuksessa sekä lisätyssä todellisuudessa. Perinteiset menetelmät ovat kyenneet hyviin tuloksiin yksinertaisissa tilanteissa, mutta seurannan heikentyminen liike-epäterävyyden ja suurien valotuksen vaihteluiden kaltaisten tekijöiden myötä motivoi vakaampien sovellusten kehittämistä.

Syväoppiminen on kerännyt viime aikoina mielenkiintoa, sillä se on osoittanut lupaavia tuloksia kuvaluokittelussa. Myös kohteenseurantaa on pidetty mahdollisena syvien arkkitehtuurien sovelluskohteena, minkä innoittamana tämä työ tutkii kohteenseurantaan käytettyjä syviä neuroverkkoja ja perusteita niiden käytölle. Syviin neuroverkkoihin perustuvia seurantasovelluksia on tutkittu, koska syvät verkot oppivat eristämään hierarkisia piirteitä. Näiden yleisten piirteiden käyttö nähdään mahdollisena ratkaisuna haastavien olosuhteiden tuottamiin onglemiin. Syvien verkkojen on havaittu myös yleistyvän hyvin ennalta tuntemattomiin kohdeluokkiin, mikä on parannus perinteisten menetelmien vaatimaan hienosäätöön ja luokkakohtaiseen tietoon. Alueella tehdyn tutkimuksen perusteella syvät neuroverkot pystyvät kilpailemaan perinteisten seurantamenetelmien kanssa.
\end{abstractpage}


%% Sisällysluettelo
\thesistableofcontents{}

%% Symbolit ja lyhenteet
\printacronyms[include-classes=abbrev,name=Abbreviations]

%% Sivulaskurin viilausta opinnäytteen vaatimusten mukaan:
%% Aloitetaan sivunumerointi arabialaisilla numeroilla (ja jätetään
%% leipätekstin ensimmäinen sivu tyhjäksi,
\cleardoublepage{}
\storeinipagenumber{}
\pagenumbering{arabic}
\setcounter{page}{1}

%% Ensimmäinen sivu tyhjäksi
\thispagestyle{empty}

%% Content

%% Kappaleet tähän
\section{Introduction}

Object tracking is a large and actively researched sub-area of computer vision. The
main task for a tracker is to find and follow the desired subject in a sequence of
images. Object tracking is closely related to other image analysis tasks so the
implementations also share elements.

The field of image classification took a leap forward in 2012, when
Krizhevsky et.\ al.\ presented record performance in the ImageNet-classification
challenge using a convolutional network. Previous work had dismissed the network
type as unfit for the task.~\cite{NIPS_IMAGENET} Since then, research has shifted
to using convolutional networks as they have several clear advantages over
other network types when used on picture analysis.
\authorcomment{go into benefits and/or give a source for the claim}

With the adoption of convolutional networks, much of the research revolves around
deep neural networks. They consist of visible input and output layers with several
so-called hidden layers in between them. The training of deep neural networks
requires a large amount of training data and their development has been made easier
by an increase in the size of applicaple datasets.

This thesis will present the architectures and principles currently used in deep
neural networks tailored to object tracking tasks. The practices behind training
and evaluating such networks are also introduced.

\section{Conclusions}

Summarize the current state of object tracking with \ac{dnn}s with possibly some
insight to future developments.


\clearpage
%% Lähdeluettelo

\thesisbibliography{}
\bibliography{kandi}

\begin{appendices}

\begin{otherlanguage}{finnish}
	\clearpage
\section*{Yhteenveto}
Kohteenseuranta videossa on aktiivisesti tutkittu konenäön osa-alue, jossa tavoitteena
on seurata osoitettua kohdetta videossa. Seurantaa ei tule sekoittaa monista älypuhelimistakin
löytyvään tunnistukseen, jossa tarkoitus on vain löytää tietyn kohdeluokan edustajat
kuvasta. Sovelluksia koohteenseurannalla muun muassa videovalvonnassa, koneen ja ihmisen
välisessä viestinnässä sekä lisätyssä todellisuudessa. Luonteeltaan kohteenseurannalla on
yhtäläisyyksiä kuva-analyysisovelluksiin ja myös käytetyt menetelmät ovat osittain samoja.
Viime vuosina kohteenseurantaan onkin sovellettu myös syviä neuroverkkoja niiden menestyttyä
kuvaluokittelussa. Tämän työ perehtyy syväoppimiseen, kohteenseurantaan sekä siihen
käytettyihin syviin neuroverkkoihin. Lisäksi työssä käsitellään tehtävään tarkoitettujen
verkkojen vertailuperiaatteita ja käytettyjä aineistoja. Keskeisimmät tutkimuskysymykset
ovat seuraavat: millaisia syviä neuroverkkorakenteita on käytetty kohteenseurantaan, kuinka
syvien verkkojen käyttö auttaa kyseisissä verkoissa ja mitä heikkouksia syvillä rakenteilla
on seurannan kannalta. Työ on luonteeltaan puhdas kirjallisuuskatsaus.

Keinotekoinen neuroverkko on löyhästi aivojen toimintaperiaatteiden mukaan rakennettu
koneoppimismalli, joka mallinnetaan yleisimmin syötteinä, tulosteina ja niitä yhdistävänä
sarjana kerroksia. Kerrokset koostuvat neuroneista, joilla on sarja painotettuja syötteitä,
syötteiden summalla syötettävä aktivaatiofunktio ja mahdollinen poikkeutusarvo. Kohteenseurannassa
tyypillinen neuronityyppi on tasasuunnattu lineaariyksikkö (rectified linear unit), joka
toteuttaa aktivaatiofunktiota $g (z) = \max\{0,z\}$. Se tuottaa hyöydyllisen epälineaarisen
muunnoksen syötteistä, mutta sen yleistyminen ja optimoinnin keveys ovat silti vertailukelpoisia
lineaaristen mallien kanssa. Syötteet ja tulosteet voidaan mallintaa vektoreina ja painot sekä
poikkeutukset matriiseina.

Syvä neuroverkko tarkoittaa tyypillisesti neuroverkkoa, jossa välikerroksina on monta
piilotettua kerrosta. Koko verkkoa siis opetetaan tuottamaan syötteistä haluttu tuloste
välittämättä yksittäisen välikerroksen tulosteista. Yleisin kohteenseurannassa käytetty
verkkotyyppi on eteenpäin syöttävä eli niin sanottu monikerrosperseptroni (multilayer
perceptron). Eteenpäin syöttävässä verkossa tieto kulkee aina kerroksesta seuraavaan,
eikä niiden välillä ole esimerkiksi takaisinkytkentöjä.

Verkon painot alustetaan tyypillisesti pienillä satunnaisluvuilla ja poikkeutusarvot nollaan
tai pieniin satunnaislukuihin. Yksinkertaisimmillaan koneoppiminen toteutetaan syöttämällä
verkolle tunnettua dataa ja optimoimalla sen painoja vertaamalla tulostetta haluttuun arvoon.
Tulosteen eroa haluttuun arvoon mallinnetaan tappiofunktiolla, jonka gradientti lasketaan
koko verkolle esimerkiksi taaksepäin syöttämällä. Menetelmässä virhearvo syötetään takaisin
pitkin kerroksia ja jokaiselle neuronille asetetaan kontribuutioarvo. Tätä arvoa käytetään
gradientin määrittämiseksi painoille ja kutakin painoa siirretään hieman vastakkaiseen suuntaan
tappiofunktion minimoimiseksi. Prosessia toistetaan uusilla koulutusaineistoilla, kunnes
koulutuksen katsotaan olevan valmis.

Konvoluutiota suoran matriisikertolaskun sijaan vähintään yhdessä kerroksessaan käyttävää
neuroverkkoa kutsutaan konvoluutioneuroverkoksi. Konvoluutio on kahdelle funktiolle suoritettava
operaatio, joka kuvataan yleisesti niiden pistetulojen integraalina toisen funktion
siirtämisen funktiona. Kuva-analyysissä konvoluutio tapahtuu käsittelemällä syötekuvaa
siirrettävällä kernelillä eli matriisilla painoarvoja. Kukin neuroni käsittelee vain
kernelinsä rajoittamaa osaa syötteestä, mutta konvoluutiokerrosten ketjuttamisen myötä
syvemmät kerrokset ovat epäsuorassa yhteydessä koko verkon syötteeseen tai suurimpaan
osaan siitä.

Konvoluutioverkon kerros muodostuu kolmesta vaiheesta: konvoluutiosta, tunnistusksesta ja
keräämisestä. Ne voidaan mallintaa verkkoon erillisinä kerroksina, mikä mahdollistaa
modulaarisen rakenteen. Konvoluutiovaiheessa syöte käsitellään kerneliä liikuttaen.
Tunnistusvaiheessa suoritetaan useita konvoluutioita rinnakkain ja kukin niistä syötetään
lineaariseen aktivaatiofunktioon. Nämä tulosteet syötetään tunnistusvaiheessa epälineaariseen
aktivaatiofunktioon ja keräämisvaiheessa lähekkäiset tunnistuksen tulosteet yhdistetään
keräysfunktiolla.

Kohteenseurantaa videossa on tutkittu vuosikymmeniä. Tyypillisesti seuranta tapahtuu videosta,
jonka ensimmäisessä kuvassa on osoitettu haluttu seurannan kohde, mutta joissain tapauksissa
sovelluksen on löydettävä se kuvauksen perusteella. Tehtävää ei olla laajasta tutkimuksesta
huolimatta ratkaistu, ja seurannan epäonnistumista aiheuttavat edelleen hankalat olosuhteet,
kuten liikeepäterävyys tai kohteen hetkellinen poistuminen kuvasta.

Merkittävät varhaiset sovellukset hyödynsivät kohteesta otettuja kuvasarjoja kohteen etsimiseksi
tai esittivät kohteen sen rajat määrittävänä käyränä. Modernit seurantamenetelmät voidaan jakaa
karkeasti generatiivisiin ja erotteleviin, mutta niiden yhdistelmiä on myös tutkittu. Generatiiviset
sovellukset etsivät kuvasta parhaiten kohteesta muodostettua mallia vastaavia alueita ja erottelevat
toteuttavat seurannan erottamalla kohteen taustasta esimerkiksi soveltamalla luokitteluverkkoa
pienempiin osiin kuvasta.

Koulutukseen käytetyt aineistot ovat yhtä tärkeitä kuin itse verkon rakenne. Seurantaverkkoa
kehitettäessä on tavallista esikouluttaa ominaisuuksia erottelevat kerrokset yksi kerrallaan
esimerkiksi luokitteluun tarkoitettujen käsin merkittyjen tai hakusanalla kuvapalveluista
ladattujen kuva-aineistojen avulla parempien tulosten takaamiseksi. Yksi merkittävimmistä
käytetyistä aineistoista on vuosittaisiin ImageNet Large Scale Visual Recognition Challenge
-haasteisiin päivitettävä kuvakanta.

Yksittäisten kuvien lisäksi koulutukseen käytetään kohteenseurantaan keskittyneitä
videoaineistoja, joilla voidaan arvioida verkon ehdottamaa kohteen sijaintia todelliseen.
Merkittäviä seurantavideoita sisältäviä aineistoja ovat TB-100 ja sen alajoukko TB-50 sekä
vuoteen 2012 järjestetyn Visual Object Tracking challenge (VOT) -kilpailun aineistot.

Kehitettyjä seurantasovelluksia arvioidaan vertaamalla niiden tuloksia aiemmin julkaistuihin
samaa aineistoa käyttämällä. Yksi käytetyistä aineistoista on Visual Tracker Benchmarkin kokoama
TB-50, johon on julkaistu sen tekohetkellä merkittävien vapaasti saatavilla olleiden sovellusten
tulokset vertailua varten. Myös VOT:n kaltaisten kilpailujen tuloksia ja aineistoja voidaan
käyttää myöhempien menetelmien arviointiin.

Viimeaikaiset syviä neuroverkkoja hyödyntävät kohteenseurantasovellukset käyttävät
tyypillisesti konvoluutioneuroverkkoa piirteiden etsimiseen kuvasta, jolloin varsinainen
paikannus voidaan toteuttaa esimerkiksi binäärisenä luokitteluna kohteen ja taustan
välillä. Tavanomaiset käsin määritettyjä malleja käyttävät menetelmät kykenevät seurantaan
hyvin, mikäli videossa ei esiinny hankalia olosuhteita. Syvien verkkojen koulutuksessa
oppimien yleisten piirteiden toivotaan poistavan luokkakohtaisen tiedon tarve menetelmää
suunniteltaessa, sillä yleisten piirteiden avulla verkko voi luoda kohteesta mallin
tuntematta sitä ennakkoon. Verkon painoja voidaan myös säätää seurannan aikana automaattisesti,
mikäli kohteen ulkonäössä havaitaan merkittävä muutos.

Tuoreessa tutkimuksessa on kiinnitetty huomiota syvässä verkossa esiintyvien piirteiden
erilaisiin ominaisuuksiin käsiteltävän tason sijainnista riippuen. Loppupään tasot
tuottavat korkean piirteitä, joista on erityisesti hyötyä kohteen luokittelussa. Aiemmilta
tasoilta voidaan saada tietoa esimerkiksi kohteen tekstuurista tai muista yksilöllisistä
ominaisuuksista, jotka ovat hyödyllisiä seurattavan kohteen erottamisessa mahdollisista
muista samankaltaisista kappaleista. Eritasoisia piirteitä voidaan hyödyntää esimerkiksi
haarauttamalla verkko yhteisiä piirteitä eristävien tasojen jälkeen ja käyttämällä
jalostettuja piirrekarttoja lopulliseen paikallistamiseen seurannan tilasta riippuen.

Julkaistun tutkimuksen perusteella syviä neuroverkkoja hyödyntämällä voidaan kehittää
kilpailukykyisiä kohteenseurantasovelluksia ja esimerkiksi DeepTrackin osoitettiin
kykenevän muita julkaisuhetkellä hallinneita seuraimia parempiin tuloksiin. Syviä
verkkoja on kuitenkin sovellettu kohteenseurantaan vasta verrattain lyhyen ajan, eikä
vertailuja niiden sovellusten välillä ole juurikaan tehty. On siis vaikea tehdä johtopäätöksiä
yksittäisten syvien seurainten paremmuudesta. Suurimpana haasteena on tällä hetkellä
menetelmien huomattavan suuret tehovaatimukset, sillä moni niistä kykeni vain muutaman
kuvan käsittelyyn sekunnissa. Harva toteutus on kuitenkaan ollut erityisesti optimoitu,
joten etenkin saatavilla olevan laskentatehon kasvaessa yhä useammilla menetelmillä voidaan
saavuttaa myös reaaliaikaiseen seurantaan tarvittava käsittelynopeus.

\renewcommand{\thesubsection}{\oldsubection}

\end{otherlanguage}

\end{appendices}

\end{document}
