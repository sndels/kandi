\documentclass[english,12pt,a4paper,pdftex,elec,utf8]{aaltothesis}

%% Kirjoita y.o. \documentclass optioiksi
%% korkeakoulusi näistä: arts, biz, chem, elec, eng, sci
%% editorisi käyttämä merkkikoodaustapa: utf8, latin1

%% Käytä näitä, jos kirjoitat englanniksi. Katso englanninokset tiedostosta
%% thesistemplate.tex.
%\documentclass[english,12pt,a4paper,pdftex,elec,utf8]{aaltothesis}
%\documentclass[english,12pt,a4paper,dvips]{aaltothesis}

\usepackage[english,finnish]{babel}
\usepackage[fixlanguage]{babelbib}
\usepackage{cite}
\usepackage[section]{placeins} %prevent figures from floating allover
\usepackage{cleveref} % for cref
\usepackage{booktabs} % for midrule etc.
\usepackage{multirow} % for multirow table
\usepackage{siunitx} % for \ang
\usepackage{float}
\usepackage{subfig} % for subfigures
\usepackage[toc,page]{appendix}
\usepackage[yyyymmdd,hhmmss]{datetime}

\usepackage{acro} % for acronyms
\acsetup{
  only-used   = true
}

\DeclareAcronym{HHI}{
  short = HHI ,
  long  = Human-Human Interaction ,
  class = acronym
}

\DeclareAcronym{HRI}{
  short = HRI ,
  long  = Human-Robot Interaction ,
  class = acronym
}


%% Aloita uudet kappaleet uudelta sivulta
\usepackage{titlesec}
\newcommand{\sectionbreak}{\clearpage}
\usepackage{graphicx}

\usepackage[numbers]{natbib}

\selectbiblanguage{english}
\bibliographystyle{babunsrt}

%% Matematiikan fontteja, symboleja ja muotoiluja lisää, näitä tarvitaan usein 
\usepackage{amsfonts,amssymb,amsbsy,amsmath}

%% Saat pdf-tiedoston viittaukset ja linkit kuntoon seuraavalla paketilla.
\usepackage[pdfa]{hyperref}
\hypersetup{pdfpagemode=UseNone, pdfstartview=FitH,
  colorlinks=true,urlcolor=red,linkcolor=blue,citecolor=black,
  pdftitle={Default Title, Modify},pdfauthor={Your Name},
  pdfkeywords={Modify keywords}}

%% Shows comments in boxes
%\newcommand{\authorcomment}[1]{\fbox{\begin{minipage}{\textwidth}Comment by author:\\#1\end{minipage}}}
%\newcommand{\advisorcomment}[1]{\fbox{\begin{minipage}{\textwidth}Comment by advisor:\\#1\end{minipage}}}

%% Hides comments
\newcommand{\authorcomment}[1]{}
\newcommand{\advisorcomment}[1]{}

%% Kaikki mikä paperille tulostuu, on tämän jälkeen
\begin{document}

%% Kansilehti
%% Korjaa vastaamaan korkeakouluasi, jos automaattisesti asetettu nimi on 
%% virheellinen 
%%
%% Change the school field to specify your school if the automatically 
%% set name is wrong
% \university{aalto-yliopisto}
% \school{Sähkötekniikan korkeakoulu}

%% VAIN DI/M.Sc.- JA LISENSIAATINTYÖLLE: valitse laitos,
%% professuuri ja sen professuurikoodi.
%%
%\department{Radiotieteen ja -tekniikan laitos}
%\professorship{Piiriteoria}
%%

%% Valitse yksi näistä kolmesta
%%
\univdegree{BSc}
%\univdegree{MSc}
%\univdegree{Lic}

%% Oma nimi
%%
\author{Santeri Salmijärvi}

%% Opinnäytteen otsikko tulee tähän ja uudelleen englannin- tai 
%% ruostinkielisen abstraktin yhteydessä. Älä tavuta otsikkoa ja
%% vältä liian pitkää otsikkotekstiä. Jos latex ryhmittelee otsikon
%% huonosti, voit joutua pakottamaan rivinvaihdon \\ kontrollimerkillä.
%% Muista että otsikkoja ei tavuteta! 
%% Jos otsikossa on ja-sana, se ei jää rivin viimeiseksi sanaksi 
%% vaan aloittaa uuden rivin.
%% 
\thesistitle{Object tracking with deep neural networks}

\place{Espoo}

%% Kandidaatintyön päivämäärä on sen esityspäivämäärä! 
%% 
\date{31.8.2017}

\supervisor{D.Sc. (Tech) Pekka Forsman}
\advisor{M.Sc. (Tech) Mikko Vihlman}

%% Aaltologo: syntaksi:
%% \uselogo{aaltoRed|aaltoBlue|aaltoYellow|aaltoGray|aaltoGrayScale}{?|!|''}
%% Logon kieli on sama kuin dokumentin kieli
%%
\uselogo{aaltoRed}{''}

%% Tehdään kansilehti
%%
\makecoverpage{}


%% Tiivistelmä
\keywords{keywords in english}
\degreeprogram{Bachelor's Program in Electrical Engineering}

\begin{abstractpage}[english]
Object tracking is a subsection of computer vision, where a target is followed through
a video. The problem has been researched for decades and it's use cases include
surveillance human-computer interaction and augmented reality. Traditional methods
have provided good results in simple conditions but lost accuracy due to distractors
like motion blur and drastic illumination changes motivates the development of more
robust trackers.

Recently, deep learning has garnered interest as it has shown promising results in
image classification. Trackers based on deep neural networks have been researched
for their ability to utilize hierarchies of features learned from training. Generic
sets of features are seen as a possible way of tackling the difficulties of tracking
in challenging conditions. Trackers built on deep neural networks have also shown
good generalization when subjected to new target classes, which improves upon the
traditional methods' requirement of hand tuning and domain specific knowledge. Based
on the research done in the field, deep neural networks are capable of results
comparable to or even better than those of traditional methods.
\end{abstractpage}

\newpage
\thesistitle{Kohteenseuranta syvillä neuroverkoilla}
\keywords{avainsanat suomeksi}
\degreeprogram{Sähkötekniikan kandidaattiohjelma}
\supervisor{TkT Pekka Forsman}
\advisor{DI Mikko Vihlman}

\begin{abstractpage}[finnish]
Kohteenseuranta on konenäön osa-alue, jossa kohdetta seurataan läpi videon. Ongelmaa
on tutkittu vuosikymmeniä ja sille on käyttökohteita muun muassa valvontasovelluksissa,
tietokoneen ja ihmisen vuorovaikutuksessa sekä lisätyssä todellisuudessa. Perinteiset
menetelmät ovat kyenneet hyviin tuloksiin yksinertaisissa tilanteissa, mutta seurannan
heikentyminen liike-epäterävyyden ja suurien valotuksen vaihteluiden kaltaisten
tekijöiden myötä motivoi vakaampien sovellusten kehittämistä.

Viime vuosina syväoppiminen on kerännyt huomiota näytettyään lupaavia tuloksia
kuvanluokittelusovelluksissa. Syviin neuroverkkoihin perustuvia seurantasovelluksia
on tutkittu niiden opetuksesta oppimien hierarkisten piirteiden ansiosta. Nämä yleiset
piirteet nähdään mahdollisena tapana ratkaista hankalien olosuhteiden tuomat haasteet
seurannassa. Syvillä verkoilla toimivat sovellukset ovat myös osoittaneet yleistyvänsä
hyvin uusiin kohdeluokkiin, mikä on parannus perinteisten menetelmien kaipaamaan
hienosäätöön ja luokkakohtaiseen tietoon. Alalla tehdyn tutkimuksen perusteella
syvät neuroverkot pystyvät perinteisten menetelmien kanssa vertailukelpoisiin tai
jopa parempiin tuloksiin.
\end{abstractpage}


%% Sisällysluettelo
\thesistableofcontents{}

%% Symbolit ja lyhenteet
\mysection{Abbreviations}
\printacronyms[include-classes=acronym,name=Acronyms]


%% Sivulaskurin viilausta opinnäytteen vaatimusten mukaan:
%% Aloitetaan sivunumerointi arabialaisilla numeroilla (ja jätetään
%% leipätekstin ensimmäinen sivu tyhjäksi,
\cleardoublepage{}
\storeinipagenumber{}
\pagenumbering{arabic}
\setcounter{page}{1}

%% Ensimmäinen sivu tyhjäksi
\thispagestyle{empty}

%% Content

%% Kappaleet tähän
\section{Introduction}

Object tracking is a large and actively researched sub-area of computer vision. The
main task for a tracker is to find and follow the desired subject in a sequence of
images. Object tracking is closely related to other image analysis tasks so the
implementations also share elements.

The field of image classification took a leap forward in 2012, when
Krizhevsky et.\ al.\ presented record performance in the ImageNet-classification
challenge using a convolutional network. Previous work had dismissed the network
type as unfit for the task.~\cite{NIPS_IMAGENET} Since then, research has shifted
to using convolutional networks as they have several clear advantages over
other network types when used on picture analysis.
\authorcomment{go into benefits and/or give a source for the claim}

With the adoption of convolutional networks, much of the research revolves around
deep neural networks. They consist of visible input and output layers with several
so-called hidden layers in between them. The training of deep neural networks
requires a large amount of training data and their development has been made easier
by an increase in the size of applicaple datasets.

This thesis will present the architectures and principles currently used in deep
neural networks tailored to object tracking tasks. The practices behind training
and evaluating such networks are also introduced.

\section{Conclusions}

This thesis first presented the basic theory of object tracking and \ac{dnn}s to provide basis for understanding the motivations in applying \ac{dnn} architectures to tracking. Several novel and successful trackers were then reviewed with an emphasis on their architectures and the reasoning used in design. Common advantages and drawbacks compared to traditional tracking methods were also noted.

\ac{dnn}s are seen as a potential alternative to hand-crafted low-level models in object tracking. The traditional methods work well in well controlled environments but can have difficulties tracking in challenging conditions like partial occlusion of the target or motion blur. The hiearchical generic features a \ac{dnn} learns are more resilient against such challenges which is a common motivation for developing trackers that use \ac{dnn}s. Learning feature representations from training is also seen as a potential alternative to hand-tuning a traditional model that typically requires domain specific knowledge.~\cite{DLT}

The research of \ac{dnn} based object tracking is still young and trackers have been proposed with emphasis on different challenges. Better generalization and being less affected by challenging conditions are common motivations for choosing deep features over traditional hand-crafted ones and trackers using \ac{dnn}s have shown competitive performance when put against traditional trackers \cite{DEEPTRACK}. However, most \ac{dnn} trackers require pre-training and real-time performance is still difficult to attain even with the advance in computing power.


\clearpage
%% Lähdeluettelo

\thesisbibliography{}
\bibliography{kandi}

\begin{appendices}

\begin{otherlanguage}{finnish}
	\section{Finnish summary - Suomenkielinen tiivistelmä}

\clearpage

\setcounter{subsection}{-1}
\let\oldsubsection=\thesubsection
\renewcommand{\thesubsection}{\thesection}

\subsection{Kohteenseuranta syvillä neuroverkoilla}

Pitkä tiivistelmä suomeksi (3 sivua)

\renewcommand{\thesubsection}{\oldsubsection}

\end{otherlanguage}

\end{appendices}

\end{document}
